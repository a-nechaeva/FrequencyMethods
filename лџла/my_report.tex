\documentclass[a5paper, 10pt]{article}

% Текст
\usepackage[utf8]{inputenc} % UTF-8 кодировка
\usepackage[russian]{babel} % Русский язык
\usepackage{indentfirst} % красная строка в первом параграфе в главе
% Отображение страниц
\usepackage{geometry} % размеры листа и отступов
\usepackage{listings}
\usepackage{color}

\geometry{
	left=12mm,
	top=25mm,
	right=15mm,
	bottom=17mm,
	marginparsep=0mm,
	marginparwidth=0mm,
	headheight=10mm,
	headsep=7mm,
	nofoot}
\usepackage{afterpage,fancyhdr} % настройка колонтитулов
\pagestyle{fancy}
\fancypagestyle{style}{ % создание нового стиля style
	\fancyhf{} % очистка колонтитулов
	\fancyhead[LO, RE]{Контрольная работа} % название документа наверху
	\fancyhead[RO, LE]{Вариант 8} % название section наверху
	\fancyfoot[RO, LE]{\thepage} % номер страницы справа внизу на нечетных и слева внизу на четных
	\renewcommand{\headrulewidth}{0.25pt} % толщина линии сверху
	\renewcommand{\footrulewidth}{0pt} % толцина линии снизу
}
\fancypagestyle{plain}{ % создание нового стиля plain -- полностью пустого
	\fancyhf{}
	\renewcommand{\headrulewidth}{0pt}
}
\fancypagestyle{title}{ % создание нового стиля title -- для титульной страницы
	\fancyhf{}
	\fancyhead[C]{{\footnotesize
			Министерство образования и науки Российской Федерации\\
			Федеральное государственное автономное образовательное учреждение высшего образования
	}}
	\fancyfoot[C]{{\large 
			Санкт-Петербург, 2024
	}}
	\renewcommand{\headrulewidth}{0pt}
}

% Математика
\usepackage{amsmath, amsfonts, amssymb, amsthm} % Набор пакетов для математических текстов
%\usepackage{dmvnbase} % мехматовский пакет latex-сокращений
\usepackage{cancel} % зачеркивание для сокращений
% Рисунки и фигуры
\usepackage[pdftex]{graphicx} % вставка рисунков
\usepackage{wrapfig, subcaption} % вставка фигур, обтекая текст
\usepackage{caption} % для настройки подписей
\captionsetup{figurewithin=none,labelsep=period, font={small,it}} % настройка подписей к рисункам
% Рисование
\usepackage{tikz} % рисование
\usepackage{circuitikz}
\usepackage{pgfplots} % графики
% Таблицы
\usepackage{multirow} % объединение строк
\usepackage{multicol} % объединение столбцов
% Остальное
\usepackage[unicode, pdftex]{hyperref} % гиперссылки
\usepackage{enumitem} % нормальное оформление списков
\setlist{itemsep=0.15cm,topsep=0.15cm,parsep=1pt} % настройки списков
% Теоремы, леммы, определения...
\theoremstyle{definition}
\newtheorem{Def}{Определение}
\newtheorem*{Axiom}{Аксиома}
\theoremstyle{plain}
\newtheorem{Th}{Теорема}
\newtheorem{Lem}{Лемма}
\newtheorem{Cor}{Следствие}
\newtheorem{Ex}{Пример}
\theoremstyle{remark}
\newtheorem*{Note}{Замечание}
\newtheorem*{Solution}{Решение}
\newtheorem*{Proof}{Доказательство}
% Свои команды
\newcommand{\comb}[1]{\left[\hspace{-4pt}\begin{array}{l}#1\end{array}\right.\hspace{-5pt} } % совокупность уравнений
% Титульный лист
\usepackage{csvsimple-l3}
\newcommand*{\titlePage}{
	\thispagestyle{title}
	\begingroup
	\begin{center}
		%		{\footnotesize
			%			Министерство образования и науки Российской Федерации\\
			%			Федеральное государственное автономное образовательное учреждение высшего образования
			%		}
		%		
		\vspace*{6ex}
		
		{\small
			САНКТ-ПЕТЕРБУРГСКИЙ НАЦИОНАЛЬНЫЙ ИССЛЕДОВАТЕЛЬСКИЙ УНИВЕРСИТЕТ ИТМО	
		}
		
		\vspace*{2ex}
		
		{\normalsize
			Факультет систем управления и робототехники
		}
		
		\vspace*{15ex}
		
		{\Large \bfseries 
			Контрольная работа
		}
\vspace*{2ex}
	{\Large \bfseries 
			
Вариант 8
		}
\vspace*{2ex}
		
		{\normalsize
			по дисциплине Частотные методы
		}

	\end{center}
	\vspace*{20ex}
	\begin{flushright}
		{\large 
			\underline{Выполнила}: студентка гр. \textbf{R3238}\\
			\begin{flushright}
				\textbf{Нечаева А. А.}\\
			\end{flushright}
		}
		
		\vspace*{5ex}
		
		{\large 
			\underline{Преподаватель}: \textit{Перегудин Алексей Алексеевич}
		}
	\end{flushright}	
	\newpage
	\setcounter{page}{1}
	\endgroup}

\begin{document}
	\titlePage
	\pagestyle{style}
\newpage

\section{Задание}

\textit{Разложить в ряд Фурье функцию, определить значения в точках разрыва:}

\begin{equation}
f(t) = 
\begin{cases}
1, & -\pi \leq t \leq 0,\\
3, & 0 < t \leq \pi
\end{cases}
\end{equation}

\subsection{Разложение в ряд Фурье}

Функция $f(t)$ удовлетворяет условию теоремы Дирихле о разложении периодической функции в ряд Фурье.

\begin{equation}
f(t) \sim \frac{a_0}{2} + \sum \limits_{n=1}^{\infty} \left( a_n \cos (nt) + b_n \sin (nt) \right),
\end{equation}

где $a_0$, $a_n$, $b_n$ -- коэффициенты Фурье:

\begin{equation}
a_0 = \frac{1}{\pi} \int \limits_{-\pi}^{\pi} f(t) dt,
\end{equation}

\begin{equation}
a_n = \frac{1}{\pi} \int \limits_{-\pi}^{\pi} f(t) \cos (nt) dt,
\end{equation}


\begin{equation}
b_n = \frac{1}{\pi} \int \limits_{-\pi}^{\pi} f(t) \sin (nt) dt,
\end{equation}

Вычислим значения коэффициентов:

\begin{equation}
a_0 = \frac{1}{\pi} \int \limits_{-\pi}^{\pi} f(t) dt = \frac{1}{\pi} \left( \int \limits_{-\pi}^{0} dt  + 3 \int \limits_{0}^{\pi} dt \right) = \frac{1}{\pi} \left( \pi  + 3 \pi \right) = 4
\end{equation}

\begin{multline}
a_n = \frac{1}{\pi} \int \limits_{-\pi}^{\pi} f(t) \cos (nt) dt = \frac{1}{\pi} \left( \int \limits_{-\pi}^{0} \cos (nt) dt  + 3 \int \limits_{0}^{\pi} \cos (nt) dt \right) = \\ =  \frac{1}{n\pi} \left( \left. \sin (nt) \right|_{-\pi}^{0} + 3 \left. \sin (nt) \right|_{0}^{\pi} \right) = \frac{1}{n\pi} \left( 0 + 3 \cdot 0 \right) = 0
\end{multline}


\begin{multline}
b_n = \frac{1}{\pi} \int \limits_{-\pi}^{\pi} f(t) \sin (nt) dt = \frac{1}{\pi} \left( \int \limits_{-\pi}^{0} \sin (nt) dt  + 3 \int \limits_{0}^{\pi} \sin (nt) dt \right) = \\ =  -\frac{1}{n\pi} \left( \left. \cos (nt) \right|_{-\pi}^{0} + 3 \left. \cos (nt) \right|_{0}^{\pi} \right) = -\frac{1}{n\pi} \left( 1 - (-1)^n + 3 \cdot \left((-1)^n - 1 \right) \right) = \\ = -\frac{1}{n\pi} \left( 2 \cdot \left((-1)^n - 1 \right) \right) = \frac{2 - 2(-1)^n}{n \pi}
\end{multline}

В итоге, запишем, полученное разложение в ряд Фурье:

\begin{equation}
f(t) \sim 2+ \frac{2}{\pi}\sum \limits_{n=1}^{\infty} \left( \frac{1 - (-1)^n}{n} \sin (nt) \right)
\end{equation}


\subsection{Нахождение значений в точках разрыва}
Согласно теореме Дирихле: \textit{если функция имеет ограниченную вариацию (конечное число точек строгого экстремума и точек разрыва первого рода), то в точках непрерывности ряд сходится к функции поточечно, а в точках разрыва сходится }\textbf{к среднему арифметическому значений двух концов}.\\

Точка разрыва первого рода $t = 0$, значение в точках разрыва\\
 $$\phi (0) = \frac{1 + 3}{2} = 2.$$








\newpage
\section{Задание}

\textit{Найти образ Фурье от функции}

\begin{equation}
f(t) = 
\begin{cases}
1 - e^{1 + t}, &  t \leq -1,\\
1 - e^{1 - t}, &  t \geq 1, \\
0, else
\end{cases}
\end{equation}

\subsection{Нахождение образа Фурье от функции}

Фурье-образ функции $f(t)$ будем находить по формуле:

\begin{equation}
\hat{f} (\omega) = \frac{1}{\sqrt{2 \pi}} \int \limits_{-\infty}^{\infty} f(t) e^{-i \omega t} dt
\end{equation}

\begin{multline}
\hat{f} (\omega) = \frac{1}{\sqrt{2 \pi}} \int \limits_{-\infty}^{\infty} f(t) e^{-i \omega t} dt =\\= \frac{1}{\sqrt{2 \pi}} \left( \int \limits_{-\infty}^{-1} (1 - e^{1 + t}) e^{-i \omega t} dt + \int \limits_{1}^{\infty} (1 - e^{1 - t}) e^{-i \omega t} dt + \int \limits_{-1}^{1} 0 \cdot e^{-i \omega t} dt \right) =\\=
\frac{1}{\sqrt{2 \pi}} \left( \int \limits_{-\infty}^{-1} (1 - e^{1 + t}) e^{-i \omega t} dt + \int \limits_{1}^{\infty} (1 - e^{1 - t}) e^{-i \omega t} dt \right) = \\ = \frac{1}{\sqrt{2 \pi}} \left( \int \limits_{-\infty}^{-1}  e^{-i \omega t} - e^{1 + t -i \omega t} dt + \int \limits_{1}^{\infty}  e^{-i \omega t} - e^{1 - t -i \omega t} dt \right) \fbox{=}
\end{multline}


\begin{equation}
\int \limits_{-\infty}^{-1} e^{1 + t -i \omega t} dt = e \int \limits_{-\infty}^{-1} e^{ t (1 -i \omega)} dt = \left. \frac{e^{1 + t (1 -i \omega)}}{1 - i \omega} \right|_{-\infty}^{-1} = \frac{e^{1 - (1 -i \omega)}}{1 - i \omega} - \frac{e^{1 + \infty (1 -i \omega)}}{1 - i \omega}
\end{equation}


\begin{equation}
\int e^{1 - t -i \omega t} dt = e \int e^{- t (1 + i \omega)} dt = -\frac{e}{1 + i \omega}  e^{- t (1 +i \omega)} + C
\end{equation}


\begin{equation}
\int \limits_{-\infty}^{-1}  e^{-i \omega t} dt  = \left. \frac{i}{\omega} e^{-i \omega t} \right|_{-\infty}^{-1} = \frac{i}{\omega} e^{i \omega} - \frac{i}{\omega} e^{i \omega \infty}
\end{equation}


\begin{equation}
\int \limits_{1}^{\infty}  e^{-i \omega t} dt  = \left. \frac{i}{\omega} e^{-i \omega t} \right|_{1}^{\infty} = \frac{i}{\omega} e^{-i \omega \infty} - \frac{i}{\omega} e^{-i \omega }
\end{equation}


\end{document}













